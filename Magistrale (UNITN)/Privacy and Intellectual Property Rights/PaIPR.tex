\documentclass[a4paper, 10pt, titlepage]{article}
\usepackage[T1]{fontenc}
\usepackage[utf8]{inputenc}

\begin{document}
\title{Privacy and Intellectual Property Rights}
\author{Edoardo Righi}
\maketitle
\section{Lecture 1}
Attending students will have to submit a short paper for the exam.
During the course, students will have to provide three short summaries of the course contents. Summary will be handled using the link provided in the main page of this course. The deadlines for handling the summaries are:
\begin{itemize}
\item 30/10/2019 (content of lectures until 24/10/2019)
\item 5/12/2019 (content of lectures between 31/10/2019 and 28/11/2019)
\item 31/12/2019 (content of lectures between 4/12/2019 and 19/12/2099)
\end{itemize}

\section{Lecture 2}
Law is based on the values of a society. Sometimes law reflects it, others it tries to shape it. In this last case, the lawmaking is inspired by the most prestigious inclinations; for example, the french after the french revolution, or the US in the last years, shaped the law of a lot of other countries. This is more recurring for nations that have similarities or deals (like italy and germany before WW2).

\section{Lecture 3}
The \textit{rationality principle} was based on God's words first, then, after the enlightment, on the words of notorious and influencial personalities, like the Greek philosophers. Rational in law means that there can not be any contraddiction in its statements.

Two distinctions:
\begin{itemize}
\item \textbf{Natural Law}: based on the core concepts of property and contract.
\item \textbf{Positive Law}: from Latin, meaning \textbf{placed}, \textbf{created}.
\end{itemize}
The property also seen as a way to asses social stature, and linked to the concept of liberty.
Also there are two types of property, in contrast:
\begin{itemize}
\item \textbf{real property} (in rem right): property of things
\item \textbf{intellectual property}: property of ideas
\end{itemize}

\section{Lecture 4}
Legal issues in CS:
\begin{itemize}
\item copyright
\item patents
\item trademarks
\end{itemize}
The copyright was created to protect the publisher. 

The \textit{copyright law} stated that the exclusive right was given in order to make society better, by disseminating knowledge. But how to encourage learning giving to the author the power to exclude you from learning? The solution is to limit this control, for example by making it temporary.

In the UK the \textbf{stationers' company}, a gild established by Queen Mary in 1557, composed of booksellers, publisers and more, worked as a monopoly; this ended with the enlightment, when LETTERATI (quelli che ne sanno) asked for the access to culture for all people, giving the momentum for the creation of a public domain.

\section{Lecture 5}
A copyright has value where it is accepted. What if I have a copyright in a nation and I cross the border? Obviously if there are not regulations, the copyright is useless. That is why an important point was the \textit{globalization} of copyright law. The agreements between different nations were formalised during the \textbf{Berne Convention} (Berne Convention for the Protection of Literary and Artistic Works), in 1886. US joined it only in 1988, when they started to become exporters of patents.

What about the copyright on code? At first it was possible only on source code, because the compiled code needed the aid of machines to be understood. \medskip \\
In Berne Convention there is no:
\begin{itemize}
\item authorship
\item originability
\end{itemize} \medskip
EU specific law for database copyright is called \textit{Sui Generis Right}.

\end{document}